% coding:utf-8

\documentclass[a4paper,11pt]{article}

\usepackage[utf8]{inputenc} 
\usepackage[T1]{fontenc} 
\usepackage{textcomp} 
%\usepackage{lmodern} 
\usepackage{graphics}
\usepackage{graphicx}
\usepackage[ngermanb]{babel}		
\usepackage{amsmath}
\usepackage{makeidx}

\title{DSVB - Testataufgabe: Goertzel Algorithmus}
\author{
    Roger Waltenspül \and Daniel Stadelmann
}


\pagestyle{headings}

\begin{document}

\maketitle
\author

\subsection{Allgemeine Notation}
Matrizen werden entweder mit grossen Buchstaben bezeichnet oder mit ihren Elementen in eckigen Klammern ausgeschrieben. \newline
\begin{center}$ A = [a_{jk}]$\end{center}

\subsection{Dimension}
Die Dimension wird mit Anzahl Zeilen $\times$ Anzahl Spalten angegeben.\newline
z.B. eine $2 \times 3$~Matrix \newline
\[A = [a_{jk}]=\left( \begin{array}{ccc}
a & b & c \\
d & e & f  \end{array} \right)\]

\subsection{Transponieren einer Matrix}
toDo

\subsection{Symmetrien}
\begin{itemize}
\item Falls $A^T = A$ dann ist A \textbf{symmetrisch}
\item Falls $A^T = -A$ dann ist A \textbf{schief-symmetrisch}
\end{itemize}


\subsection{Rechenregeln}
\begin{itemize}
\item $A + B = B + A$
\item 
\end{itemize}



\end{document}
