% coding:utf-8

\documentclass[a4paper,11pt]{article}

\usepackage[utf8]{inputenc} 
\usepackage[T1]{fontenc} 
\usepackage{textcomp} 
%\usepackage{lmodern} 
\usepackage{graphics}
\usepackage{graphicx}
\usepackage[ngermanb]{babel}		
\usepackage{amsmath}
\usepackage{makeidx}
\usepackage{listings}
\usepackage{color}

\definecolor{dkgreen}{rgb}{0, 0.6, 0}
\definecolor{gray}{rgb}{0.5, 0.5, 0.5}
\definecolor{mauve}{rgb}{0.58, 0, 0.82}

% used for code insertions
\lstset{frame=tb,
  language=C,
  aboveskip=3mm,
  belowskip=3mm,
  showstringspaces=false,
  columns=flexible,
  basicstyle={\small\ttfamily},
  numbers=none,
  numberstyle=\tiny\color{gray},
  keywordstyle=\color{blue},
  commentstyle=\color{dkgreen},
  stringstyle=\color{mauve},
  breaklines=true,
  breakatwhitespace=true,
  tabsize=3
}

\title{DSVB - Testataufgabe: Goertzel Algorithmus}
\author{
    Roger Waltenspül \and Daniel Stadelmann
}

\pagestyle{headings}

\begin{document}

\maketitle
\author
\thispagestyle{empty}

\newpage
\pagenumbering{arabic}
\section{Q1 - Q15 Format}
Im Normalfall wird beim umrechnen ins Q15-Format ein Bitshift um 15 Stellen durchgeführt. 
\begin{equation}\label{eq:q15format}
	(a_{Q_{15}} * b_{Q_{15}}) \gg 15 = C_{Q_{15}}
\end{equation}
Dies kann jedoch zu Problemen führen, falls $a_{Q_{15}}$ oder $b_{Q_{15}}$ den Wert $2^{-15}$ annehmen. Wenn einer dieser Werte unser Koeffizient ist und wir garantieren können, dass dieser nie diesen Wert annimmt ist das Shiften um 15 Stellen nach rechts zulässig.
In unserem Code des Goertzel-Algorithmus wird jedoch lediglich um 14 Stellen geshiftet (siehe Code unten).
\begin{lstlisting}
void goertzel_filter_v0 ( short int * delay, short int input, short int coefficient )
{
    long product;

    product = ( (long) delay[1] * coefficient ) >> 14;
    ...
\end{lstlisting}

Dies weil \ldots

\section{Q2 - v1 Goertzel Algorithmus}

\section{Q3 - Signal Power Berechnung}

\section{Q4 - Power Calculation Methoden im Vergleich}

\begin{tabular}{lll}
\hline
   & calculation method v0 & calculation method v1 \\
\hline
computational effort    & blabla    & blabla  \\
numerical robustness & blablabla  & blablabla  \\
\hline
\end{tabular}

\end{document}
